\documentclass{beamer}

\setbeamertemplate{navigation symbols}{}

\usetheme{AnnArbor}
\usecolortheme{beaver}
\setbeamercolor{frametitle}{fg=black,bg=white!90!black}
\setbeamercolor{title}{fg=white,bg=red!90!black}

\setbeamertemplate{enumerate items}[default]
\setbeamercolor*{enumerate item}{fg=black}
%\setbeamercolor{titlelike}{parent=structure,bg=cyan}

\usepackage{amsfonts, amssymb, amsmath}

\newcounter{saveenumi}
\newcommand{\seti}{\setcounter{saveenumi}{\value{enumi}}}
\newcommand{\conti}{\setcounter{enumi}{\value{saveenumi}}}

\resetcounteronoverlays{saveenumi}


\begin{document}
\title{Introduction to Mathematical Analysis HW 6}
\author{Lee Young Jae\\ 2016-11988} 
\date{\today} 

\frame{\titlepage} 

% \frame{\frametitle{Table of contents}\tableofcontents} 


\section{Exercise 1} 
\frame{\frametitle{Problem 1}
Let $\Omega$ be a set.
$\mathcal{A} \subset \mathcal{P}(\Omega)$ is called $\sigma$-\textit{algebra} in $\Omega$, if
\begin{enumerate}[(i)]
\item $\Omega \in \mathcal{A}$,
\item $A \in \mathcal{A}$ implies $A^c := \Omega \backslash A \in \mathcal{A}$,
\item $A_i \in \mathcal{A}, i \in \mathbb{N}$, implies $\bigcup\limits_{i\in\mathbb{N}} A_i \in \mathcal{A}$.
\end{enumerate}
Show the following:
% \setbeamertemplate{enumerate items}[default]
\begin{enumerate}[(1)]
\item
The $\sigma$-ring $\mathcal{M}(m)$ of 7.1.11, $m$ defined as in (7.11) of the lecture, is a $\sigma$-algebra in $\mathbb{R}^d$.

\item
Let $\mu$ be a regular and finite set function on $\mathcal{E}$.
The $\mu$-zero sets are defined as the sets $N \subset \mathbb{R}^d$ for which $\mu^*(N) = 0$.
Show that the $\mu$-zero sets form a $\sigma$-ring.
\end{enumerate}
}

\subsection{Solution of 1-1}


\frame{ 
\frametitle{Solution of 1-1}
% Let $A, A_n \in \mathcal{M}(m)$ and verify the three conditions of $\sigma$-algebra.
\begin{enumerate}[(i)]
\item
$I_n = (-n,n) \in \mathcal{E} \subset \mathcal{M}_F(\mu)$ for any $n$.
Therefore, $\mathbb{R} = \bigcup\limits_{n\in\mathbb{N}} I_n \in \mathcal{M}(m)$

\item
Let $A = \bigcup\limits_{n\in\mathbb{N}}A_n \in \mathcal{M}(m), A_n \in \mathcal{M}_F(m)$.
Since $I_m \cap A_n \in \mathcal{M}_F(m)$ and $\mu^*(I_m \cap A) < \infty$, $I_m \cap A \in \mathcal{M}_F(m) \Rightarrow I_n \backslash A = I_n \backslash I_n \cap A \in \mathcal{M}_F(\mu)$ since $\mathcal{M}_F(m)$ is a ring.
Therefore, $A^c = \bigcup\limits_{n=1}^\infty (I_n\backslash A) \in \mathcal{M}(m)$.
% Then, $A^c = \mathbb{R} \backslash A = $
%$A^c = \mathbb{R} \backslash A \in \mathcal{M}(m)$ since $\mathcal{M}(m)$ is a $\sigma$-ring.

\item
%$\bigcup\limits_{n\in\mathbb{N}} A_n \in \mathcal{M}(m)$ since $\mathcal{M}(m)$ is a $\sigma$-ring.
Let $A_n = \bigcup\limits_{k=1}^\infty A_{nk}, A_{nk} \in \mathcal{M}_F(m)$ for each $n$.
Then, $\bigcup\limits_{n\in\mathbb{N}} A_n = \bigcup\limits_{(n,k)\in\mathbb{N}\times\mathbb{N}} A_{nk} \in \mathcal{M}(m)$.

\end{enumerate}
}

\subsection{Solution of 1-2}
\frame{ 
\frametitle{Solution of 1-2}
\begin{enumerate}
\item 
Let $A, B$ be $\mu$-zero sets.
For any $\epsilon > 0$ there exists $A_n, B_n \in \mathcal{E}$ such that
$$\sum_{n=1}^\infty \mu(A_n), \sum_{n=1}^\infty \mu(B_n) < \epsilon / 2 \text{ and } \bigcup_{n=1}^\infty A_n \subset A, \bigcup_{n=1}^\infty B_n \subset B$$ so that
$$\sum_{n=1}^\infty \mu(A_n \cup B_n) < \epsilon \text{ and } \bigcup_{n=1}^\infty (A_n \cup B_n) \subset (A \cup B).$$
Therefore, $\mu^*(A \cup B) < \epsilon$ for any $\epsilon > 0$, i.e., $\mu^*(A \cup B) = 0$.

\item
Let $A, B$ be $\mu$-zero sets.
Then, $\mu^*(A\backslash B) \leq \mu^*(A) = 0$ since every $(A_n) \subset \mathcal{E}$ such that $\bigcup\limits_{n=1}^\infty A_n \supset A$ satisfies $A_n \supset A\backslash B$.\\
Therefore, $A\backslash B$ is also $\mu$-zero set.
\seti
\end{enumerate}
}

\frame{
\begin{enumerate}
\conti
\item
Let $A_n$ be $\mu$-zero sets, and let $(A_{nm})_{m\geq 0} \in \mathcal{E}$ satisfies
$$\sum_{m=1}^\infty \mu(A_{nm}) < \epsilon/2^n \text{ and }\bigcup_{m=1}^\infty A_{nm} \supset A_n$$
for each $n$.
Then, similar to 1,
$$\sum_{n,m =1}^{\infty}\mu(A_{nm}) < \epsilon \text{ and } \bigcup_{n,m =1}^{\infty} A_{nm} \subset A.$$
Therefore, $\mu^*(A) < \epsilon$ for any $\epsilon > 0$, i.e., $\mu^*(A) = 0$.
\end{enumerate}
By 1 and 2, $\mu$-zero sets form a ring and by 3 it form a $\sigma$-ring.
}



\section{Exercise 2} 
\frame{\frametitle{Problem 2}
Let $\Omega$ be a set.
\begin{enumerate}[(i)]
\item
Let $I$ be an arbitrary index set (not necessarily countable) and for any $i \in I$, let $\mathcal{A}_i$ be a $\sigma$-algebra in $\Omega$. Show that
$$
\bigcap_{i\in I} \mathcal{A}_i := \{ A \subset \Omega\ |\ A \in \mathcal{A}_i,\ \forall i \in I \}
$$

\item
Let $\phi \neq \mathcal{A}_0 \subset \mathcal{P}(\Omega)$ be a collection of subsets of $\Omega$.
Then show that
$$
\sigma(\mathcal{A}_0) := \bigcup\limits_{\substack{ \mathcal{B} \ \sigma-\text{algebra} \\ \text{in } \Omega,\ \mathcal{A}_0 \subset \mathcal{B}}}\mathcal{B}
$$
is well defined and is the smallest $\sigma$-algebra in $\Omega$ containing $\mathcal{A}_0$.
($\sigma(\mathcal{A}_0)$ is called the $\sigma$-algebra generated by $\mathcal{A}_0$).
\end{enumerate}
}

\subsection{Solution of 2-1}
\frame{
\frametitle{Solution of 2-1}
Check three conditions in Exercise 1.
\begin{enumerate}[(i)]
\item Since $\Omega \in \mathcal{A}_i$ for each $i \in I$, $\Omega \in \bigcap\limits_{i\in I} \mathcal{A}_i$.

\item
If $A \in \bigcap\limits_{i\in I}\mathcal{A}_i$, then $A^c \in A_i$ for each $i \in I$.
Therefore, $A^c \in \bigcap\limits_{i\in I} \mathcal{A}_i$

\item
If $A_1, A_2, \cdots \in \bigcap_{i\in I} \mathcal{A}_i$ for each $i\in I$, then
$\bigcup\limits_{n=1}^\infty A_n \in \mathcal{A}_i$ for each $i \in I$.
Therefore, $\bigcup\limits_{n=1}^\infty A_n \in \bigcap\limits_{i\in I} \mathcal{A}_i$.
\end{enumerate}
Therefore, $\bigcap\limits_{i\in I} \mathcal{A}_i$ is again a $\sigma$-algebra in $\Omega$.
}

\subsection{Solution of 2-2}
\frame{
\frametitle{Solution of 2-2}
\begin{enumerate}
\item
We've showed that intersection of $\sigma$-algebras are also $\sigma$-algebra.
To show that it is well-defined, we only have to show that there is at least one $\mathcal{B}$ such that $\mathcal{B}$ is a $\sigma$-algebra in $\Omega$ and $\mathcal{A}_0 \subset \mathcal{B}$.
Such $\mathcal{B}$ always exists since $\mathcal{B} = \mathcal{P}(\Omega)$ satisfies the above.

\item Let $\sigma(\mathcal{A}_0) \not\subset \mathfrak{M}$ but $\mathfrak{M}$ is a $\sigma$-algebra containing $\Omega$.
Then, by definition of $\sigma(\mathcal{A}_0)$, $\sigma(\mathcal{A}_0) \subset \mathfrak{M}$. ($\Rightarrow\!\Leftarrow$)
Therefore, $\sigma(\mathcal{A}_0)$ is minimal $\sigma$-algebra containing $\mathcal{A}_0$.
\end{enumerate}
}

\section{Exercise 3} 
\frame{\frametitle{Problem 3}
Let $\phi$ be a positive set function on a ring $\mathcal{R}$.
Suppose that $\phi$ is countably additive.
Show that $\phi$ is continuous from above:
$$
A, A_n \in \mathcal{R}, \phi(A_n) < \infty, n \in \mathbb{N} \text{ and } A_n \searrow A \Rightarrow \lim_{n\rightarrow\infty} \phi(A_n) = \phi(A)
$$
\textit{Notation:} For sets $A, A_1, A_2, A_3, \cdots$ the notation $A_n \searrow A$ means that $A_1 \supset A_2 \supset A_3 \cdots$ and $\cup_{n\in\mathbb{N}} A_n = A$.
}

\subsection{Solution of 3}
\frame{
\frametitle{Solution of 3}
Let $A, A_n \in \mathcal{R}, \phi(A_n) < \infty, n \in \mathbb{N}$ and $A_n \searrow A$.
Let $B_n = A_n - A_{n+1}$.
Then, $A, B_1, B_2 \cdots$ are disjoint so that $A_1\backslash A = \bigcup\limits_{n=1}^\infty B_n$ and
% $$ \phi(A_1) = \phi\left(\bigcup_{n=0}^\infty B_n\right) = \sum_{n=0}^\infty \phi(B_n) = \sum_{n=1}^\infty \phi(B_n) + \lim_{n\rightarrow\infty}\phi(A_{n}). $$
% and
$$
\phi(A_1\backslash A) = \phi\left(\bigcup_{n=1}^\infty B_n\right) = \sum_{n=1}^\infty \phi(B_n) = \sum_{n=1}^\infty \phi(B_n).
$$
Moreover, for each $m \in \mathbb{N}$ it is true that
$$
\phi(A_1\backslash A) = \phi\left(\bigcup_{n=1}^{m-1}B_n \cup (A_m\backslash A)\right) = \sum_{n=1}^{m-1} \phi(B_n) + \phi(A_m\backslash A).
$$
Take limits both sides to get $\lim\limits_{m\rightarrow\infty} \phi(A_m\backslash A) = \lim\limits_{m\rightarrow\infty} \phi(A_m) - \phi(A) = 0$.
Therefore, $\lim\limits_{n\rightarrow\infty}\phi(A_n) = \phi(A)$.

}

\section{Exercise 4}
\frame{\frametitle{Problem 4}
Complete the proof of Proposition 7.1.17 of the lecture notes.
}

\subsection{Solution of 4}
\frame{
\frametitle{Solution of 4}
It only remains to show: if $A \in \mathcal{M}(\mu), \mu(A) = \infty,$ and $\epsilon > 0$, then there exists $G \supset A$, $G$ open with $\mu(G\backslash A)<\epsilon$.

% Since $S \in \mathcal{M}_F(\mu) \Rightarrow \mu^*(S) < \infty$.

Let $A = \bigcup\limits_{n=1}^\infty A_n, A_n \in \mathcal{M}_F(\mu)$.
Then, $\mu^*(A_n) = \mu(A_n) < \infty$ for each $n$.

% Let $A = \bigcup\limits_{n=1}^\infty A_n, A_n \in \mathcal{M}_F(\mu)$ and $\mu(A_n) < \infty$ for each $A_n$.
Therefore for each $n$ there exists $A_{nm} \in \mathcal{E}, m \geq 1$ with $\bigcup\limits_{m\geq 1} A_{nm} \supset A_m$ and
$$
\mu^*(A_{n}) + \frac{\epsilon}{2^{n+1}} > \sum_{m\geq 1} \mu(A_{nm}).
$$
}
\frame{
Following the proof of the lecture note, since $\mu(A_n) < \infty$,
%$$ \mu(\bigcup_{m\geq 1} A_{nm}) \leq \sum_{m\geq 1}\mu(A_{nm}) < \mu^*(A_{n}) +\epsilon/2^n $$
 $$\mu\left( \bigcup_{m\geq 1}A_{nm}\backslash A_n \right) < \epsilon/2^n
 \text{ and therefore } \mu\left(\bigcup_{n,m\geq 1} A_{nm} \backslash A\right) < \epsilon,$$
since
$$
\sum_{n=1}^{\infty}\mu\left(\bigcup_{m\geq 1}A_{nm}\right) \geq \mu\left(\bigcup_{m\geq 1}A_{nm}\right).
$$
We may hence get $G := \bigcup\limits_{n,m\geq 1}A_{nm}$.
}

\end{document}


